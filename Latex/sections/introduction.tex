\chapter{Introduction}

\section*{Project Context}

Although global alcohol consumption rates have fallen since 2010, in 2016 43\% (2.348 billion) of the global population had consumed alcohol in the past 12 months, with the European region having the highest rate of consumption at 59.9\% (449 million) of the total population  \cite{WHOGlobalStatusReportFull}.
Alcohol is among the leading causes of premature deaths worldwide, with alcohol being attributed to 3 million deaths per year \cite{WHOGlobalStatusReportFull}. Alcohol is also the most significant risk factor for those aged 15-49 \cite{lancetAlcoholEditorial}. This is also an age range where people are most economically active, hence the economic consequences are significant.

Alcohol is well known to have both immediate and long-term adverse health effects \cite{fullAlcoholHarms}. There is some conflicting literature on whether there is a safe level of alcohol consumption, with some studies showing that there is no safe level of alcohol consumption and others showing that participants with light alcohol consumption had lower risk of cancer and death than those who abstained \cite{alcoholRiskThresholds, noDrinkvsSomeDrink}. 

%Alcohol consumption statistics in the British Isles are collected in two regions, where Scotland collects data separately to England and Wales. \todo{UK stats} Maybe not to add england and wales stats

The \ac{AHP} describes a phenomenon in which those with higher \ac{SES} experience less overall alcohol-associated harm even when consuming greater amounts \cite{unravellingAHP}. The \ac{AHP} can be demonstrated both globally and locally, hence is a phenomenon which occurs across scales \cite{WHOGlobalStatusReportFull, lancetAlcoholEditorial, englandAlcohol2021, scotlandAlcohol2022, ahp2016}. 
Although the underlying causes of the \ac{AHP} are still unknown, some hypotheses have been proposed as potential explanations, such as low \ac{SES} individuals having less access to healthcare and heavy drinkers becoming more prone to falling down in their \ac{SES} \cite{ahpInterventions}. The conventional approach to addressing public health issues is presupposed on changing the actions of individuals, which negates the social and economic context in which people live \cite{csHealthDisparities, sdhInterventions, FCTorigin}. These approaches have so far come up against policy resistance, suggesting an approach considering a wider societal context may be useful in understanding and solving the \ac{AHP}. 
\ac{FCT} is a social theory which recognises the health-related ramifications of differential access to physical and social resources, and has been proposed as a potential social theory to explain the \ac{AHP} \cite{FCTorigin, Boyd2021}.

\ac{ABM} is a technique used successfully in public health modelling, most notably during the COVID-19 pandemic \cite{covidABM}. Some \ac{ABM}s have been used in the context of alcohol in public health \cite{scopingReview}. This dissertation project will look to build an \ac{ABM} through the lens of \ac{FCT} to see whether the \ac{AHP} can be explained through \ac{FCT}.

\chapter{Development \& Methods}


\begin{figure}[H]
\begin{center}
\begin{tikzpicture}[domain=0:4]

  \draw[->] (-1, 0) -- (4.2,0) node[right] {$c$};
  \draw[->] (0,-1) -- (0,4.2) node[above] {$AR(c)$};

  \draw[color=black] (1,0) node[below] {$T$};
  
  \draw[scale=1, domain=0:3, smooth, variable=\x, black, postaction={decorate, decoration={markings, mark=at position 0.99 with {\node[above, sloped] {$AR(c) = \beta (c-T)$};}}}] plot ({\x+1}, {\x});

\end{tikzpicture}
\caption{Absolute risk as a function of consumption \cite{}}
\label{fig:absRisk}
\end{center}
\end{figure}


\todo{reference this from Jens paper}
The absolute risk of alcohol consumption is used to model mortality in the model, and is explained through equation \ref{eq:absRisk}

\begin{equation}
  AR(c)=
  \begin{cases}
    0 & \text{if } c \leq T \\
    \beta (c-T) & \text{otherwise }
  \end{cases}
  \label{eq:absRisk}
\end{equation}